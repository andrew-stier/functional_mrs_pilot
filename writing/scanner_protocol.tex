\documentclass[10pt,a4paper]{scanner_protocol}

% Change the page layout if you need to
\geometry{left=1cm,right=9cm,marginparwidth=6.8cm,marginparsep=1.2cm,top=1cm,bottom=1cm}

% Change the font if you want to.

% If using pdflatex:
\usepackage[utf8]{inputenc}
\usepackage[T1]{fontenc}
\usepackage[default]{lato}
\usepackage{ragged2e}
\usepackage{hyperref}
\usepackage{amssymb}

% If using xelatex or lualatex:
% \setmainfont{Lato}

% Change the colours if you want to
\definecolor{Bright}{HTML}{58318f}
\definecolor{Green}{HTML}{319866}
\definecolor{Black}{HTML}{111111}
\definecolor{LightGrey}{HTML}{515c50}
\colorlet{heading}{Green}
\colorlet{accent}{Bright}
\colorlet{emphasis}{Black}
\colorlet{body}{LightGrey}

% Change the bullets for itemize and rating marker
% for \risk if you want to
\renewcommand{\itemmarker}{{\small\textbullet}}
\renewcommand{\ratingmarker}{\faSpinner}

%% sample.bib contains your publications
\addbibresource{sample.bib}

\begin{document}
\name{Functional MRS Pilot Protocol}
%\tagline{This is the simple version of a complex protocol}
\made{May 29, 2019}
\logo{6.5cm}{"logo"}


\docinfo{%
  % can add more \addedtopeople
  \madeby{Andrew Stier}{andrewstier@uchicago.edu}{May 29, 2019}
  %\addedto{John Smith}{abv1@uni.ac.uk}{October 3, 2017}
}
\purpose{
    The goal of this pilot is to examine for the first time the ability of MRS to capture functional fluctuations in metabolite levels in response to naturalistic stimuli. This will help to bridge invasive research in non-human species that examines how neurotransmitter levels impact local brain networks with research linking changes in a global network structure to various psychopathologies in humans. We expect this to have broad implications for pharmacological treatment of common psychopathologies and increase the effectiveness of biologically based diagnoses of psychopathology.
} % add a short description of the purpose for this protocol


%% Make the header extend all the way to the right, if you want.
\begin{fullwidth}
\makeheader
\end{fullwidth}

%% Provide the file name containing the sidebar contents as an optional parameter to \need.
%% You can always just use \marginpar{...} if you do
%% not need to align the top of the contents to any
%% \need title in the "main" bar.
%\need[scanner_protocol_info]{Protocol \hfill 85 minutes}
%\marginpar{%\need{Warnings}
%\warningsigns{\faFlask}{Chemical}{The H$_2$SO$_4$ is bad for you}
%\warningsigns{\insectssymbol}{Bugs}{The bugs will be attracted to the H$_2$O}



\need{Notes}
\href{https://bit.ly/2HIPzOw}{\textcolor{cyan}{IRB Link}}

\divider

CC400 Coordinates (MNI Space):
    \begin{itemize}
        \item Left Cerebellum \hfill -8.9, -70.5, -32.6
        \item Left Occipital Pole \hfill -24.2, -96.1, 6.2
        \item Right Paracingulate \hfill 7.6, 44.7, 16.1
        \item Right Supramarginal \hfill 62.2, -23.3, 29.4
    \end{itemize}
    
\divider

Participants will watch a movie throughout the session. In order to aid in between subjects comparisons, the movie will start with the same triggers as each of the MRS and fMRI scans and will be stopped at the end of each scan.

 \divider
 
 We aim to collect data from 8 participants
 
 \divider

%\need{People to contat}

%Jame and Sam (about chemicals)


}

\marginpar{%\need{Warnings}
%\warningsigns{\faFlask}{Chemical}{The H$_2$SO$_4$ is bad for you}
%\warningsigns{\insectssymbol}{Bugs}{The bugs will be attracted to the H$_2$O}



\need{Notes}
\href{https://bit.ly/2HIPzOw}{\textcolor{cyan}{IRB Link}}

\divider

CC400 Coordinates (MNI Space):
    \begin{itemize}
        \item Left Cerebellum \hfill -8.9, -70.5, -32.6
        \item Left Occipital Pole \hfill -24.2, -96.1, 6.2
        \item Right Paracingulate \hfill 7.6, 44.7, 16.1
        \item Right Supramarginal \hfill 62.2, -23.3, 29.4
    \end{itemize}
    
\divider

Participants will watch a movie throughout the session. In order to aid in between subjects comparisons, the movie will start with the same triggers as each of the MRS and fMRI scans and will be stopped at the end of each scan.

 \divider
 
 We aim to collect data from 8 participants
 
 \divider

%\need{People to contat}

%Jame and Sam (about chemicals)


}
\vspace{-125px}
{\color{heading}\LARGE\bfseries\MakeUppercase{Protocol \hfill 85 minutes}}\\[-1ex]%
{\color{heading}\rule{\linewidth}{2pt}\par}\medskip
\step{1}{Initial Setup}{21}
\begin{itemize}
	\item[$\Box$] Participant Loading \hfill 10 minutes
	\item[$\Box$] Localizer \hfill 1 minute
	\item[$\Box$] Structural Scan \hfill 10 minutes
\end{itemize}
\divider

\step{2}{MRS Scan 1: Left Cerebellum}{12.5}
\begin{itemize}
	\item[$\Box$] Locate MRS Voxel \hfill 30 seconds
	\item[$\Box$] Shimming \hfill 2 minutes
	\item[$\Box$] MRS Scan \hfill 10 minutes
\end{itemize}
\divider


\step{3}{MRS Scan 2: Left Occipital Pole (Visual Cortex)}{12.5}
\begin{itemize}
	\item[$\Box$] Locate MRS Voxel \hfill 30 seconds
	\item[$\Box$] Shimming \hfill 2 minutes
	\item[$\Box$] MRS Scan \hfill 10 minutes
\end{itemize}
\divider

\step{4}{MRS Scan 3: Right Paracingulate Gyrus (Right Anterior Cingulate Cortex/ vmPFC/ACC)}{12.5}
\begin{itemize}
	\item[$\Box$] Locate MRS Voxel \hfill 30 seconds
	\item[$\Box$] Shimming \hfill 2 minutes
	\item[$\Box$] MRS Scan \hfill 10 minutes
\end{itemize}
\divider

\step{5}{MRS Scan 4: Right Supramarginal Gyrus}{12.5}
\begin{itemize}
	\item[$\Box$] Locate MRS Voxel \hfill 30 seconds
	\item[$\Box$] Shimming \hfill 2 minutes
	\item[$\Box$] MRS Scan \hfill 10 minutes
\end{itemize}
\divider

\step{6}{fMRI Scan: Whole Brain}{14}
\begin{itemize}
    \item[$\Box$] Setup Slices \hfill 2 minutes
	\item[$\Box$] Shimming \hfill 2 minutes
	\item[$\Box$] fMRI Scan \hfill 10 minutes
\end{itemize}
\divider

\clearpage

%\need[scanner_protocol_info]{Sources}

%\nocite{*}

%\printbibliography


%\divider


%% If the NEXT page doesn't start with a \need but you'd
%% still like to add a sidebar, then use this command on THIS
%% page to add it. The optional argument lets you pull up the
%% sidebar a bit so that it looks aligned with the top of the
%% main column.
% \addnextpagesidebar[-1ex]{page3sidebar}


\end{document}
